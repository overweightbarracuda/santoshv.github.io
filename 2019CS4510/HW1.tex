 \documentclass[12pt]{article}

%\usepackage{clrscode}
%\usepackage{alltt}
%\usepackage{epsfig}
\usepackage{amssymb, amsmath}
\usepackage{amsfonts}
%\usepackage{latexsym}
%\usepackage{xypic}
%\usepackage{float}    
\newcommand{\problem}[2]{\vspace{.1in} \noindent {\bf #1}. [#2 points]\\ }
\newcommand{\problemname}[2]{\vspace{.1in} \noindent {\bf Problem #1}. [#2 points] \\ }
\newcommand{\problemnameno}[2]{\vspace{.1in} \noindent {\bf Problem #1}. [#2 points] No explanation necessary. \\ }
\parindent0em
%\newcommand{\s}{\hspace*{1em}}          
%\renewcommand{\Comment}[1]{}

%\pagestyle{myheadings}         
%\markboth{\hfill Name: \hspace{2in} \hfill}{Name:  \hspace{2in}  \hfill} 
%\markboth{ Name: \hspace{2in} \hfill}

\title{CS4510: HW1}


\date{Due: Aug 26th before 3pm on Gradescope (there is a link on Canvas).}

\begin{document}

\maketitle

\problem{1. Recognition}{2}
Write a program (in your favorite language or pseudocode) that takes as input a finite binary string and outputs ``YES" if the string contains one of the following as a substring (in consecutive positions) and outputs ``NO" if it does not: $\{010, 101, 0110, 1001\}$. Prove that your program is correct on all possible input binary strings.

\problem{2. Meta-Recognition}{2}
Write a program that takes as input a finite list of binary strings and outputs a program that takes a binary string as input and outputs ``YES" if it contains a string on the list as a substring and ``NO" otherwise. 

\problem{3. Space and Time}{2}
How much space (memory) does your first program take on inputs with $n$ bits? How much time does it take as a function on $n$? Make all your assumptions explicit. You can use $O(.)$ notation. 

(Bonus) What about your second program?  


\end{document}
